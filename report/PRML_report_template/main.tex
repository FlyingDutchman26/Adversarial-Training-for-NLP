\documentclass{article}

% if you need to pass options to natbib, use, e.g.:
%     \PassOptionsToPackage{numbers, compress}{natbib}
% before loading neurips_2020

% ready for submission
% \usepackage{neurips_2020}

% to compile a preprint version, e.g., for submission to arXiv, add add the
% [preprint] option:
\usepackage[preprint]{neurips_2020}

% to compile a camera-ready version, add the [final] option, e.g.:
%     \usepackage[final]{neurips_2020}

% to avoid loading the natbib package, add option nonatbib:
%    \usepackage[nonatbib]{neurips_2020}

\usepackage[utf8]{inputenc} % allow utf-8 input
\usepackage[T1]{fontenc}    % use 8-bit T1 fonts
\usepackage{hyperref}       % hyperlinks
\usepackage{url}            % simple URL typesetting
\usepackage{booktabs}       % professional-quality tables
\usepackage{amsfonts}       % blackboard math symbols
\usepackage{nicefrac}       % compact symbols for 1/2, etc.
\usepackage{microtype}      % microtypography
\usepackage{float}
\bibliographystyle{unsrt}

\newtheorem{theorem}{Theorem}
\newtheorem{proposition}{Proposition}
\newtheorem{lemma}{Lemma}
\newtheorem{corollary}{Corollary}
\newtheorem{remark}{Remark}
\newtheorem{assumption}{Assumption}
\newtheorem{definition}{Definition}

\title{Latex Template For PRML Assignments %titile here
}

\author{% Reviese your personal information here
  Author \\ % Your name 
  Department \\ % CS 
  ID: 12345678 \\
  \texttt{author@fudan.edu.cn} \\
}

\begin{document}

\maketitle

\begin{abstract}
   The abstract must be limited to one paragraph.
\end{abstract}

\section{Introduction}
In this section, one should introduce what you have explored in this assignment.

\section{Dataset}
In this section, you should introduce the dataset \cite{DBLP:journals/corr/abs-1708-07747} and the results of your Exploratory Data Analysis (EDA).


\section{Methodology}
Introduce your methods used here.
You can arrange more than one sections to well express your methods.

\section{Results}
Report your results here.
You can arrange more than one sections to well express your results.


In most cases, figures help your paper easier to understand, insert figure like Figure \ref{fig1:example}.
\begin{figure}[H]
  \centering
  \fbox{\rule[-.5cm]{0cm}{4cm} \rule[-.5cm]{4cm}{0cm}}
   \caption{Sample figure caption.}
  \label{fig1:example} 
\end{figure}

All artwork must be neat, clean, and legible. Lines should be dark enough for
purposes of reproduction. The figure number and caption always appear after the
figure. Place one line space before the figure caption and one line space after
the figure. The figure caption should be lower case (except for first word and
proper nouns); figures are numbered consecutively.

Tables are also necessary in some cases to present your results better.
\begin{table}
  \caption{Sample table title}
  \label{sample-table}
  \centering
  \begin{tabular}{lll}
    \toprule
    \multicolumn{2}{c}{Part}                   \\
    \cmidrule(r){1-2}
    Name     & Description     & Size ($\mu$m) \\
    \midrule
    Dendrite & Input terminal  & $\sim$100     \\
    Axon     & Output terminal & $\sim$10      \\
    Soma     & Cell body       & up to $10^6$  \\
    \bottomrule
  \end{tabular}
\end{table}


\section{Conclusion}
Conclude your paper in this section.


\textbf{Notice:} This \LaTeX{} template is modified by Peng Li with template of NIPS2020. Assignments \textbf{must} be submitted with this template, and use \verb+PDFLaTex+ directly to compile your \verb+tex+ files.


More information can be found in the following site if you find troubles in compiling your \verb+tex+ files: 
\begin{center}
\url{https://neurips.cc/Conferences/2020/PaperInformation/StyleFiles}
\end{center}


\textbf{Hint:} Do not change any aspects of the formatting parameters in the style files.  In
particular, do not modify the width or length of the rectangle the text should
fit into, and do not change font sizes.


\bibliography{reference}



\end{document}